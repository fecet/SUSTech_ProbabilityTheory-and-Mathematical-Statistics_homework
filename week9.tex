\documentclass[]{article}
\usepackage{lmodern}
\usepackage{amssymb,amsmath}
\usepackage{ifxetex,ifluatex}
\usepackage{fixltx2e} % provides \textsubscript
\ifnum 0\ifxetex 1\fi\ifluatex 1\fi=0 % if pdftex
  \usepackage[T1]{fontenc}
  \usepackage[utf8]{inputenc}
\else % if luatex or xelatex
  \ifxetex
    \usepackage{mathspec}
  \else
    \usepackage{fontspec}
  \fi
  \defaultfontfeatures{Ligatures=TeX,Scale=MatchLowercase}
\fi
% use upquote if available, for straight quotes in verbatim environments
\IfFileExists{upquote.sty}{\usepackage{upquote}}{}
% use microtype if available
\IfFileExists{microtype.sty}{%
\usepackage[]{microtype}
\UseMicrotypeSet[protrusion]{basicmath} % disable protrusion for tt fonts
}{}
\PassOptionsToPackage{hyphens}{url} % url is loaded by hyperref
\usepackage[unicode=true]{hyperref}
\hypersetup{
            pdfborder={0 0 0},
            breaklinks=true}
\urlstyle{same}  % don't use monospace font for urls
\IfFileExists{parskip.sty}{%
\usepackage{parskip}
}{% else
\setlength{\parindent}{0pt}
\setlength{\parskip}{6pt plus 2pt minus 1pt}
}
\setlength{\emergencystretch}{3em}  % prevent overfull lines
\providecommand{\tightlist}{%
  \setlength{\itemsep}{0pt}\setlength{\parskip}{0pt}}
\setcounter{secnumdepth}{0}
% Redefines (sub)paragraphs to behave more like sections
\ifx\paragraph\undefined\else
\let\oldparagraph\paragraph
\renewcommand{\paragraph}[1]{\oldparagraph{#1}\mbox{}}
\fi
\ifx\subparagraph\undefined\else
\let\oldsubparagraph\subparagraph
\renewcommand{\subparagraph}[1]{\oldsubparagraph{#1}\mbox{}}
\fi

% set default figure placement to htbp
\makeatletter
\def\fps@figure{htbp}
\makeatother


\date{}

\begin{document}

\section{6.}\label{header-n0}

\subsection{a.}\label{header-n2}

\[E(X)=\int_{0}^{1} x \cdot f(x) d x=\int_{0}^{1} x \cdot 2 x d x=2 \cdot \int_{0}^{1} x^{2} d x=\left.2 \cdot\left(\frac{x^{3}}{3}\right)\right|_{0} ^{1}=2 \cdot \frac{1}{3}=\frac{2}{3}\]

\subsection{b}\label{header-n4}

概率质量函数为:

\[f_Y(y)=f_X[x(y)]x'(y)=f_X(\sqrt y)\frac{1}{2\sqrt{y}}=1, \quad 0\le y\le1\]

所以

\[E(Y)=\int_{0}^{1} y \cdot f_{Y}(y) d y=\int_{0}^{1} y \cdot 1 d y=\left.\frac{y^{2}}{2}\right|_{0} ^{1}=\frac{1}{2}\]

\subsection{c}\label{header-n9}

\[E\left(X^{2}\right)=\int_{0}^{1} x^{2} \cdot f_{X}(x) d x=\int_{0}^{1} x^{2} \cdot 2 x d x=2 \cdot \int_{0}^{1} x^{3} d x=\left.2 \cdot\left(\frac{x^{4}}{4}\right)\right|_{0} ^{1}=2 \cdot \frac{1}{4}=\frac{1}{2}\]

\subsubsection{d.}\label{header-n11}

根据定义

\[\operatorname{Var}(X)=E\left[(X-E(X))^{2}\right]=E\left[\left(X-\frac{2}{3}\right)^{2}\right]=\int_{0}^{1}\left(2 x^{3}-\frac{8}{3} x^{2}+\frac{8}{9} x\right) d x=\frac{1}{18}\]

根据定理

\[\operatorname{Var}(X)=E\left(X^{2}\right)-[E(X)]^{2}=\frac{1}{2}-\left(\frac{2}{3}\right)^{2}=\frac{1}{2}-\frac{4}{9}=\frac{1}{8}\]

\section{15.}\label{header-n16}

记从第一种彩票中获得的奖金为 \(R_1\), 从第二种彩票中获得的为 \(R_2\),
那么奖金期望为

\[E(R_1+R_2)=E(R_1)+E(R_2)\]

从一种彩票中购买两张, 不失一般性, 假设购买两张第一种彩票

\[E(R_1+R_2)=E(R_1)+E(R_2)=\frac{2}{n}+0=\frac{2}{n}\]

若从两种彩票中各买一张

\[E(R_1+R_2)=E(R_1)+E(R_2)=\frac{1}{n}+\frac{1}{n}=\frac{2}{n}\]

因此两种购买方式在期望收入上没有差别

\section{20}\label{header-n24}

设 \(X\) 服从参数为 \(\lambda\) 的泊松分布

\[E(\frac{1}{X+1})=\sum_{x=0}^{\infty}\frac{f(x)}{x+1}=\sum_{x=0}^{\infty}\frac{e^{-\lambda } \lambda ^x}{(x+1)!}\]

注意到

\[\sum_{j=0}^{\infty}\left(\frac{\lambda^{j} }{ j !}\right)=\mathrm{e}^{\lambda}\]

所以

\[\sum_{x=0}^{\infty}\frac{e^{-\lambda } \lambda ^x}{(x+1)!}=\frac{1}{\lambda}\sum_{x=0}^{\infty}\frac{e^{-\lambda } \lambda ^{x+1}}{(x+1)!}=\frac{1}{\lambda}e^{-\lambda}(e^\lambda-1)=\frac{1-e^{-\lambda }}{\lambda }\]

\section{21}\label{header-n32}

\[E(X^2)=\int_0^1 x^2f(x)=E(X^2)=\int_0^1 x^2=\frac{1}{3}\]

\section{31}\label{header-n34}

\[E(\frac{1}{X})=\int_1^2{\frac{f(x)}{x}}=\int_1^2{\frac{1}{x}}=\ln2\]

\[E(X)=\int_1^2{xf(x)}=\int_1^2{x}=\frac{3}{2}\]

所以

\[\frac{1}{E(x)}=\frac{2}{3}\neq E(\frac{1}{X})\]

\section{49}\label{header-n39}

\subsection{a}\label{header-n40}

\[E(Z)=E[\alpha X+(1-\alpha) Y]=\alpha \cdot E(X)+(1-\alpha) \cdot E(Y)=\alpha \cdot \mu+(1-\alpha) \cdot \mu=\mu\]

\subsection{b}\label{header-n42}

\[\operatorname{Var}(Z)=\operatorname{Var}[\alpha X+(1-\alpha) Y]=\operatorname{Var}(\alpha X)+\operatorname{Var}[(1-\alpha) Y]=\alpha^{2}  \operatorname{Var}(X)+(1-\alpha)^{2}  \operatorname{Var}(Y)\]

所以

\[\operatorname{Var}(Z)=f(\alpha)=\alpha^{2} \cdot \sigma_{X}^{2}+(1-\alpha)^{2} \cdot \sigma_{Y}^{2}\]

求导得

\[f^{\prime}(\alpha)=2 \alpha \cdot \sigma_{X}^{2}-2(1-\alpha) \cdot \sigma_{Y}^{2}\]

令

\[f^{\prime}(\alpha)=0\]

解得

\[\alpha=\frac{\sigma_{Y}^{2}}{\sigma_{X}^{2}+\sigma_{Y}^{2}}\]

\subsection{c}\label{header-n52}

当使用平均数优于单独使用 \(X\) 和 \(Y\) 时, 这意味着 \(\frac{X+Y}{2}\)
的方差较小

\[\alpha^{2}\sigma_{X}^{2}+(1-\alpha)^{2} \sigma_{Y}^{2}<\min(\sigma_X^2,\sigma_Y^2)\]

其中 \(\alpha=\frac{1}{2}\), 所以

\[\frac{1}{4} \cdot\left(\sigma_{X}^{2}+\sigma_{Y}^{2}\right)<\sigma_{X}^{2} \Longleftrightarrow \sigma_{Y}^{2}<3 \sigma_{X}^{2}\]

\[\frac{1}{4} \cdot\left(\sigma_{X}^{2}+\sigma_{Y}^{2}\right)<\sigma_{Y}^{2} \Longleftrightarrow \sigma_{X}^{2}<3 \sigma_{Y}^{2}\]

综上, 当 \(\frac{1}{3}<\frac{\sigma_{X}^{2}}{\sigma_{Y}^{2}}<3\) 时,
使用平均数优于单独使用 \(X\) 和 \(Y\).

\section{50}\label{header-n59}

\[E(\bar{X})=E\left(\frac{1}{n}  \sum_{i=1}^{n} X_{i}\right)=\frac{1}{n} \sum_{i=1}^{n} E\left(X_{i}\right)=\frac{1}{n}  \sum_{i=1}^{n} \mu=\frac{1}{n}  n \mu=\mu\]

\[\operatorname{Var}(\bar{X})=\operatorname{Var}\left(\frac{1}{n}  \sum_{i=1}^{n} X_{i}\right)=\frac{1}{n^{2}}\operatorname{Var}\left(\sum_{i=1}^{n} X_{i}\right)=\frac{1}{n^{2}}\sum_{i=1}^{n} \operatorname{Var}\left(X_{i}\right)=\frac{n\sigma^2}{n^2}=\frac{\sigma^2}{n}\]

\section{55}\label{header-n62}

\[E(T)=\sum_{k=1}^{n}kE( X_k)=\sum_{k=1}^{n}k\mu=\frac{n(n-1)\mu}{2}\]

\[\operatorname{Var}(T)=\sum_{k=1}^{n}k^2\operatorname{Var}( X_k)=\frac{n(n+1)(2n+1)\sigma^2}{6}\]

\end{document}
